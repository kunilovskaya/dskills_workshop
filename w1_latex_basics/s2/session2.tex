% this is source code for one of the sessions in Digital Skills for Research Workshop (EMTTI, University of Wolverhampton)
% March 2022, Maria Kunilovskaya (mkunilovskaya@gmail.com)

\documentclass[a4paper,11pt,leqno]{article}

\usepackage[utf8]{inputenc}

\usepackage{geometry}
\geometry{
	a4paper,
	total={170mm,257mm},
	left=20mm,
	top=15mm,
}
\setlength\parindent{0pt} % set all indents to 0

\usepackage{tcolorbox}
\usepackage{listings}  % a verbatim environment which can break lines unlike \verb||

%\usepackage[hidelinks]{hyperref} % no linky content is visible at all, except if hovering a mouse
\usepackage[colorlinks=true, linkcolor=blue, urlcolor=cyan, filecolor=magenta]{hyperref}

% multilang support
%--------------------------------------
\usepackage[T1,T2A]{fontenc}
\usepackage[spanish,french,russian,main=english]{babel}
%--------------------------------------


\title{Session 2. Text and Math}
\author{Digital Skills for Research}
\date{\today}

\begin{document}

\maketitle
\tableofcontents

\section{Explore the interface of your desktop \TeX~editor}

\begin{itemize}
	\item line numbers are useful for fixing errors
	\item use the toolbar to insert typical commands, if you prefer
\end{itemize}

\section{Text formatting}


\subsection{Languages}

 \href{https://www.overleaf.com/learn/latex/International_language_support}{International language support}

\begin{lstlisting}[breaklines]
\usepackage[T1, T2A]{fontenc}
\usepackage[french,russian,main=english]{babel}
\end{lstlisting}

See specific packages names for individual languages \href{https://github.com/kunilovskaya/dskills_workshop/tree/main/alphabets}{here}. \\ For example, you need \verb|\usepackage{arabtex} and \usepackage[LFE,LAE]{fontenc}| for Arabic.

\begin{description}
	\item[T1] encoding for Latin script
	\item[T2A] encoding for Cyrillic script
\end{description}

\begin{itemize}
	\item You might want to install languages to your \textbf{local} \TeX distribution on Linux with \\ \verb|sudo apt install texlive-lang-polish| \\
	or install them all \verb|sudo apt-get install -y texlive-lang-all|

	\item You might need to specifically import fonts for other languages.
	\item NB! main (or last one) in \verb|\usepackage[french,russian,main=english]{babel}| is the language for automatic structural elements (e.g. Contents)
	\item In some templates short text snippets in non-main language are inserted with
	\verb|\foreignlanguage{spanish}{...}| 
\end{itemize}

Examples in Spanish and Russian:

\begin{center}
	\begin{quote}
		\foreignlanguage{spanish}{Sección Introductoria} 
		
		\foreignlanguage{russian}{Спасибо.}
	\end{quote}
	
	Sección Introductoria
	Спасибо.
\end{center}

\subsection{Fonts and colours}

\begin{description}
	\item[font style] roman (serif), bold, italics, typewriter, sans serif, small caps respectively in \\
	\verb|\textrm,\textbf,\textit, \texttt, \textsf, \textsc|
	
	e.g. \textsc{Encyclopedia Galactica} 
	
	\texttt{Don't Panic}
	
	\item[font size] \verb|\tiny, \scriptsize, \footnotesize, \small, \normalsize, \large, \Large, \LARGE, \huge, \Huge| \\
	e.g. cf. \verb|\huge{text formatting}| and \verb|\tiny{text formatting}|: \\
	\huge{text formatting} \\
	\tiny{text formatting}
	\normalsize
	
	\item[underline] \underline{Google strikethrough!}
	
	\item[font colour] \verb|{\color{red!50!white}your_coloured_text}| \\
	 e.g. {\color{red!50!white}my pink text}
 
\end{description}


\subsection{Special characters}
The following characters are reserved to have special meaning in \LaTeX:

\$ \% \& \verb|~| \verb|^| \{ \} \_ $\backslash$ 

To print them literally you need to escape them with backslash ($\backslash$), except
\begin{itemize}
	\item for $\backslash$ itself use \verb|$\backslash$|
	\item for \verb|~| and \verb|^| Use them inside \verb|\verb||| 
\end{itemize}

\textbf{Formatting conventions:}

\begin{itemize}
	\item double quotes: tilde key (left uppermost key under Esc) without SHIFT and single quote twice $\rightarrow$ `` and ''
	\item \verb|\lq| and \verb|\rq| or tilde key + single quote $\rightarrow$ \lq~and \rq or ` and '
	\item non-breaking space (SHIFT + tilde key): \verb|Fig.~\ref{fig:logo}|
	\item dots: \verb|\ldots| $\rightarrow$ \ldots
	\item dash: \verb|--| $\rightarrow$ --
	\item superscripts and subscripts: \verb|2^2 and CO_2| $\rightarrow$ $2^2$, $CO_2$
	
	e.g. Proceedings of the 11\textsuperscript{th} conference\ldots
	
	\item accents: \verb|\^{a}, \'{o}, \v{c}, \"{e}, \~{o}| $\rightarrow$ \^{a}, \'{o}, \v{c}, \"{e}, \~{o}
	\item language-specific ligatures: \verb|\ss{}, \l{}, \L{}| $\rightarrow$ \ss{}, \l{}, \L{}
\end{itemize}

\textbf{Useful tool: \href{http://detexify.kirelabs.org/symbols.html}{Detexify} (\textcolor{red}{CLICK ME! I am linked!}) symbol table and classifier}

\section{Math}

\subsection{Mathmode}

\begin{itemize}
	\item inline: \verb|$your_formula$|
	\item on a separate line: \verb|\[your_formula\]|
	\item special environment with a counter in label: \verb|\begin{equation}\label{eq:entropy} ... \end{equation}|
\end{itemize}

This formula $f(x)=5x$ is incorporated in the body of text. 

This is the same formula on a separate line:

\[ f(x)=5x \]

This is the same formula in the environment (notice the number on the right/left!). See equation~\ref{eq:same}. You can move the number to the left by adding leqno as an optional argument to the class definition \verb|\documentclass[a4paper,11pt,leqno]{article}| 

\begin{equation}\label{eq:same}
f(x)=5x
\end{equation}

\subsection{Fractions}
\verb|\[\frac{1+\frac{4}{2}}{6} = 0,5\]|

\[\frac{1+\frac{4}{2}}{6} = 0,5\]

\subsection{Brackets}
\verb|\[ \left(2+\frac{9}{3}\right) \times 5 = 25 \]|

\[ \left(2+\frac{9}{3}\right) \times 5 = 25 \]

\[  [2+3]  \]

\verb|\[ \{2+3\}  \]|
\[ \{2+3\}  \]

\subsection{Standard functions}
\verb|$\sin x = 0$, $\cos x = 1$, $\ln x = 5$|
$\sin x = 0$, $\cos x = 1$, $\ln x = 5$

\subsection{Symbols}
\verb|$2\times 2 \ne 5$|

$2\times 2 \ne 5$

$A \cap B$, $A \cup B$

\subsection{Characters from other scripts}
\verb|$\epsilon$, $\phi$|

$\tg \Phi = 1$

$\epsilon$, $\phi$

\bigskip

There are useful tools that reduce the sufferings:

\begin{enumerate}
	\item \href{https://latexeditor.lagrida.com/}{Online Equation Editor} 
	\item \textbf{{\color{red}Copy formatted formulas from html source code}} (Inspect $\rightarrow$ Copy $\rightarrow$ Copy element):
	
	\href{https://en.wikipedia.org/wiki/Second_law_of_thermodynamics}{Second Law of Thermodynamics}
	
	\[\mathrm {d} S={\frac {\delta Q}{T}}-{\frac {1}{T}}\sum _{j}\,\Xi _{j}\,\delta \xi _{j}\]
\end{enumerate}

\section*{Task 2. Reproduce a pdf formatting, inc. formulas}
\label{task}
\addcontentsline{toc}{section}{Task 2}

\begin{tcolorbox}[width=\textwidth, colback={yellow!40!white}, title={}, colbacktitle=yellow!60!white, coltitle=black]
	\begin{itemize}
		\item Reproduce text formatting on this \href{https://github.com/kunilovskaya/dskills_workshop/blob/main/w1_latex_basics/s2/practice2.pdf}{page}
		\item Link your resulting pdf for Task 2 in the \href{https://docs.google.com/document/d/17ZBAQGBKIlO6JMwxz3LlghYq1sdsUjhHVXga46BK0kg/edit?usp=sharing}{Achievement Tracker}
	\end{itemize}
	
\end{tcolorbox}%


\end{document}