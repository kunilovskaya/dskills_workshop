% this is source code for one of the sessions in Digital Skills for Research Workshop (EMTTI, University of Wolverhampton)
% March 2022, Maria Kunilovskaya (mkunilovskaya@gmail.com)

\documentclass[a4paper,11pt]{article}

\usepackage[utf8]{inputenc}

\usepackage[colorlinks=true, linkcolor=blue, urlcolor=cyan, filecolor=magenta]{hyperref} 

\usepackage{geometry}
\geometry{
	a4paper,
	total={170mm,257mm},
	left=20mm,
	top=15mm,
}
\setlength\parindent{0pt}

% verbatim environment enhance to provide for code listings
\usepackage{listings}
\lstset{basicstyle=\ttfamily\footnotesize}
\usepackage{tcolorbox} 
\usepackage{wrapfig}

% this redifines \href which by default adds .pdf if it does not see a recorgnisable file extension, wich is the case when a folder is passed
\makeatletter
\newcommand\HREF[2]{\hyper@linkurl{#2}{#1}}
\makeatother

\usepackage{soul}  % strilethrough

\title{Session 8. Version control and collaboration: Git and GitHub}
\author{Digital Skills for Research}
\date{March 25, 2022}


\begin{document}
	
	\clearpage
	\maketitle
	\thispagestyle{empty}
	
	\tableofcontents 

\section*{}

Familiarity with Git/GitHub is required by employers in translation industry:
e.g. from a job advert for a project/product manager at Smartling:
\begin{quotation}
	\noindent Do you have Git experience? Please give us 3 examples where you used GitHub.\\
	What do you find more challenging with GitHub?
\end{quotation}

This session offers a practical and systematic explanation of core functionality of Git and GitHub. \\	

\section{Git: Keeping track of changes}

For software developers, Git allows \textbf{parallel maintenance of older and newer versions} of the product. Each released version is stored as a separate branch of the project. 

For a researcher, Git is a way to \textbf{collaborate and keep track of contributions} from participants as well as \textbf{making your work available} to other researchers. 

\bigskip

Important features:
\begin{wrapfigure}{r}{0.3333\textwidth}
	\centering
	\includegraphics[width=60mm]{pics/linux_git_creator_torvalds.jpg}
	\caption{Linus Torvalds}
\end{wrapfigure}
\begin{itemize}
	\item by Linus Torvalds (2005), a Finnish-American software engineer, who is the main developer of the Linux kernel. First Linux prototype by released in 1991 when Torvalds was 22.
	\item OSS (open-source soft)
	\item de facto standard in academia and industry
	\item Distributed Version Control System (full history of changes is kept locally and remotely)	
\end{itemize}

\subsection{Is Git installed in your OS?}

\begin{itemize}
	\item By default, Git is installed on Linux and macOS computers as a command line option.
	\item \href{https://git-scm.com/book/en/v2/Getting-Started-Installing-Git}{How to install Git on Windows}
\end{itemize}

\subsection{HowTo}

\textbf{Start tracking a \textbf{local} project folder}

\begin{enumerate}
	
	\item Go into the \HREF{./demo/}{\texttt{demo}} directory containing the project in the terminal (``Open in terminal/command line/cmd'')
	\item create README\textbf{.md} and \textbf{.}gitignore files in the root of the project
	\item \verb|git init|
	\item \verb|git add README.md .gitignore| to add \textbf{\textsc{relevant}} files
	\item \verb|git commit -m `first commit'|
\end{enumerate}

Each time you want to update the history of changes in the project, pass \verb|git add| and \verb|git commit| commands (don't forget a descriptive but short message!)

\medskip

\textcolor{red}{Boom! If you look at hidden files in the tracked folder, you will see a .git/}

\section{GitHub: Local and remote, push and pull, auth}

\subsection{Connect to remote URL at GitHub server}

Git associates a remote URL with a name, and your default remote is usually called \textcolor{red}{origin}.

You can only push to two types of URL addresses:

\begin{itemize}
	\item \textbf{(default)}: An HTTPS URL like https://[hostname]/user/repo.git 
	\item (not discussed here) An SSH URL, like git@[hostname]:user/repo.git
\end{itemize}

\textbf{Connect a tracked local project to GitHub}

\begin{itemize}
	\item Log in to your GitHub account
	\item Click the new repository button in the top-right
	\item Click the ``Create repository'' button
	\item It is easier to give it the same name as the name of folder to connect
	\item Decide whether you are ready to go public (\textcolor{red}{default!}) with your project
	\item run the following commands from the local tracked folder in terminal \\
	
	\verb|git remote add origin https://github.com/kunilovskaya/demo.git| \\
	\verb|git push -u origin master|
\end{itemize}

\textcolor{red}{master} means main branch of the project. \\
\textcolor{red}{origin master} is main branch of the remote repo (on the server).

\subsection{Auth: username and password/access token/SSH key [and a passphrase])}

Select username wisely! e.g. ssharoff, TharinduDR, ltgoslo, torvalds \\

How passwords and tokens are used: 

\begin{itemize}
	\item \textbf{website password} is required to create/delete repositories and add changes to them \textbf{in the browser}
	\item access token is needed to push local content to a remote repo with HTTPS URL \textbf{from the command line}
	\item SSH key [or a passphrase to it] is needed to push content to a remote repo with SSH URL	
\end{itemize}

See detailed and official  \href{https://docs.github.com/en/enterprise-server@3.4/get-started/getting-started-with-git/about-remote-repositories}{HowTo} \\

\textcolor{red}{NB!} Password-based authentication for Git has been removed on August 13, 2021.

\bigskip

\textbf{Permanently authenticating with Git repositories:} \\
Locally stored access keys apply to all projects.
\begin{enumerate}
	\item create a config file \\
	\begin{lstlisting}
	git config credential.helper store
	git push https://github.com/kunilovskaya/dskills_workshop.git
	\end{lstlisting}
	\item give your username and the new key to save it remotely
	\item set a longer cache timeout than the default 15 mins (e.g. 2 hours or 5 days=7200 min) to avoid accessing the txt file with the \textcolor{red}{unencrypted password stored on your local disk} each time you push
	\begin{lstlisting}
	git config --global credential.helper 'cache --timeout 7200' 
	\end{lstlisting}
	
\end{enumerate}

Instructions \href{https://stackoverflow.com/questions/8588768/how-do-i-avoid-the-specification-of-the-username-and-password-at-every-git-push}{here}

\subsection{Typical workflow}

\begin{enumerate}
	\item publish local changes in files (or \verb|git rm -r folder-name|) to the website
	\begin{itemize}
		\item \verb|git add my_file1.py my_file2.py|
		\item \verb|git commit -m `added export to tsv'|
		\item \verb|git push| (if you have set up automatic authentification, you will not be asked for username and password)
	\end{itemize}
	\item get changed made on the remote \\
	(if you know that someone might have pusshed to your repo, pull changes first to avoid conflicts)
	\begin{itemize}
		\item \verb|git pull|
	\end{itemize}
\end{enumerate}

\subsection{PyCharm-Git(Hub) intergration}

PyCharm 2021.3 (Professional Edition) \\

File $\rightarrow$ Settings $\rightarrow$ Version control $\rightarrow$ Github 

\begin{itemize}
	\item Log in via Github
	\item Log in with Token
	\item (a tickbox) clone git repo using ssh
\end{itemize}

\subsection{Overleaf-Git(Hub) intergration}

\textbf{GitHub Sync is a premium feature}

\begin{center}
	
\includegraphics[width=10cm]{pics/try_it_for_free.png}

\end{center}

 

\section{Markdown and arranging repos}

\begin{enumerate}
	\item don't push everything: \st{git add .} even if .gitignore exists
	\item don't push data and output (mind 2MB limit for one file upload)
	\item create a clear description in README (using markdown); see \href{https://github.com/kunilovskaya/dskills_workshop/blob/main/README.md}{my example}
\end{enumerate}

\textcolor{red}{markdown (.md)} is a markup language for creating formatted text using a plain-text editor.

One useful \href{https://github.com/adam-p/markdown-here/wiki/Markdown-Cheatsheet}{Markdown Cheatsheet} \\

How to format:
\begin{itemize}
	\item headings
	\item emphasis
	\item line breaks
	\item lists
	\item links
	\item tables
	\item code listings\\
	\begin{lstlisting}
	```python
		s = "Python syntax highlighting"
		print(s)
	```
	\end{lstlisting}
	\item horizontal rule
	
\end{itemize}

\section*{Task 7. Start a private repo and add `kunilovskaya' as a collaborator}
\label{task}
\addcontentsline{toc}{section}{Task 7. Creat a repo}

\begin{tcolorbox}[width=\textwidth, colback={yellow!40!white}, title={}, colbacktitle=yellow!60!white, coltitle=black]
	\begin{itemize}
		\item set up Git tracking for a local folder (with a README);
		\item push it to GitHub;  
		\item invite a collaborator;
		\item make changes on the server (in browser)
		\item pull remote changers
		\item make local changes and push them (don't forget to refresh the page!)
	\end{itemize}
	
\end{tcolorbox}%

\end{document}