\documentclass[a4paper,11pt]{article}

\usepackage{hyperref}


\usepackage{geometry}
\geometry{
	a4paper,
	total={170mm,257mm},
	left=20mm,
	top=15mm,
}

\title{Session 2. Text and math}
\author{Digital Skills for Researchers}
\date{\today}

\begin{document}

\maketitle
\tableofcontents

\section{Text formatting}

\subsection{Explore the interface of your \TeX~editor}

\subsection{Languages, fonts, colors}

font families (\textrm,\textbf,\textit, \textsl, \textsc, \texttt, \textsf, \textnormal, \underline)
font sizes (tiny, scriptsize, footnotesize, small, normalsize, large, Large, LARGE, huge, Huge)
\textsc{Encyclopedia Galactica}
\texttt{Don't Panic} 
\underline{It begins with a house.}

\section{Special characters}
The following characters are reserved to have special meaning in \LaTeX:

\$ \% \& \verb|~| \verb|^| \{ \} \_ $\backslash$ 

To print them literally you need to escape them with backslash ($\backslash$), except
\begin{itemize}
	\item for $\backslash$ itself use \verb|$\backslash$|
	\item for \verb|~| and \verb|^| Use \verb|\verb| 
\end{itemize}
 
quotes, ` \lq and ‘ \rq 
\verb|| command for inline verbatim mode

\ldots

subscripts and superscripts
Proceedings of the 11\textsuperscript{th} conference\ldots

ligatures
accents \^{}, \'{}, \v{}, \"{}, \c{}, \~{}
language-specific \ss{}, \l{}, \L{}

In writing and typography, a ligature occurs where two or more graphemes or letters are joined as a single glyph




\section{Math}


\end{document}