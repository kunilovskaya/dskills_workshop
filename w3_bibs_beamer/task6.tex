% this is source code for one of the sessions in Digital Skills for Research Workshop (EMTTI, University of Wolverhampton)
% March 2022, Maria Kunilovskaya (mkunilovskaya@gmail.com)

\documentclass[a4paper,11pt]{article}

\usepackage[utf8]{inputenc}
\usepackage[colorlinks=true, linkcolor=blue, urlcolor=cyan, filecolor=magenta]{hyperref}
\usepackage{geometry}
\geometry{
	a4paper,
	total={170mm,257mm},
	left=20mm,
	top=60mm,
}

\usepackage{graphicx}

\title{Practical Task for Session 6. \\Annotated bibliography and revision}
\author{YOUR NAME}
\date{ACTUAL DATE of completion}

\begin{document}
	
	\maketitle

\bigskip

You can use this file as a starting point to complete the task. You might want to delete the passage with the assignment below.


\begin{enumerate}
	\item As explained \href{https://www.wlv.ac.uk/lib/media/departments/lis/skills/study-guides/LS136-Guide-to-Writing-an-Annotated-Bibliography.pdf}{here}, it is a list of publications with your comments and takeaways.
	
	If you are using Mendeley:
	\begin{itemize}
		\item Link Mendeley account to your Overleaf account
		\item Get bibliographic descriptions of several items you plan to read and put them into a shared folder in Mendeley
		\item Create the project .bib from the folder contents (see Figure~\ref{fig:integrate})
		\item (optional) Can you tweak Mendeley to make use of the field \verb|annote = {These are my take-aways},| which pulls your Notes for the item in Mendeley?
	\end{itemize}
	\item In the source code, which you are encouraged to share with m.kunilovskaya@wlv.ac.uk, use your own words to comment the usage of the packages in the preamble. 
\end{enumerate}
	
\end{document}
