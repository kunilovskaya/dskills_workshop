% this is source code for one of the sessions in Digital Skills for Research Workshop (EMTTI, University of Wolverhampton)
% March 2022, Maria Kunilovskaya (mkunilovskaya@gmail.com)

\documentclass[a4paper,11pt]{article}

% custom link command
\usepackage[colorlinks=true, linkcolor=blue, urlcolor=cyan, filecolor=magenta]{hyperref} 

\usepackage{geometry}
\geometry{
	a4paper,
	total={170mm,257mm},
	left=20mm,
	top=15mm,
}
\setlength\parindent{0pt} % set all indents to 0

\usepackage{listings}  % a verbatim environment which can break lines unlike \verb||; load AFTER babel
\usepackage{tcolorbox}
\usepackage{multicol}
\usepackage{todonotes}

\usepackage{graphicx}  % to add graphics
\graphicspath{{images/}{pics/}}  % folders with .png, .jpj, .gif, .eps, .pdf
\usepackage{wrapfig} % put figure inside the text
\usepackage{caption} % automatic names for graphics; default Figure
%\captionsetup{labelsep=period} % add a dot after Figure in captions

%--------------------
% Own commands
% -------------------
\newcommand{\myLaTeX}{\LaTeX~}

\renewcommand*{\figurename}{Fig.}

\newcommand{\boxedfig}[1]{%
	\setlength{\fboxsep}{5pt}%
	\setlength{\fboxrule}{3pt}%
	\fbox{\includegraphics[width=\linewidth]{#1}}%
}

\newenvironment{hello}[1][world]{\noindent Hello #1, }{Bye now!\\} % first [] has number of arguments; second [] has the default value of the first optional argument; second argument is mandatory
\newcommand{\hi}[2][world]{\noindent Hello #1 and #2}

%Numbered environment with double counter-within

\newcounter{example}
\counterwithin*{example}{section} % asterisk/star avoids redefining theexample (second number in 1.2) in each section
\newenvironment{examples}[1][mytitle]{\refstepcounter{example}\par\medskip
	\noindent \textbf{Example~\thesection.\theexample. #1}\par \rmfamily}{\medskip}

%--------------------
% Title
% -------------------

\title{Session 4. Customisation, cross-referencing and templates}
\author{Digital Skills for Research}
\date{March 11, 2022}

\begin{document}

\maketitle
\tableofcontents

\section{Customisation and own commands}\label{sec:own}


Usually, there are many ways to skin the \hypertarget{wd:random}{cat}.

Levels of customisation: 
class files, style files, packages that provide additional commands and environments, own commands and environments + advanced \texttt{xparse} and \texttt{etoolbox} packages for writing own packages

\begin{itemize}
	\item \verb|\newcommand|: defines a new command; it is a \myLaTeX wrapper on top of \TeX~primitive (\verb|\def|) 
	\begin{itemize}
		\item Adding a space after LaTeX default logo command: \\
		\verb|\newcommand{\myLaTeX}{\LaTeX|$\sim$\}
		\item \verb|\newenvironment{hello}[1][world]{\noindent Hello #1, }{Bye now!\\}|
	\end{itemize}
	\item \verb|\renewcommand|: redefines an existing command
	\begin{itemize}
		\item \verb|\renewcommand{\harvardurl}{URL: \url}|
		\item \verb|\renewcommand{\refname}{Selected Publications 2017-2021}|
		\item \verb|\renewcommand{\figurename}{Fig.}|
		\item (re)new(ed) commands/environments can have [optional] and \{mandatory\} arguments: e.g. \verb|\newcommand{\boxedfig}[1]{...}|
		
	\end{itemize}
	\item modify the default parameters (a) globally for the whole document or (b) locally for parts of it: \\ e.g. 
	\begin{itemize}
		\item put \verb|\setlength\parindent{0pt}| in preamble to cancel all indentation (or \verb|\noindent| for local effect)
		\item \verb|\captionsetup{labelsep=period}| to use a dot (not colon) after Fig(ure) 1 in captions
		\item \verb|\thispagestyle{empty}| on any page to lose the page number (see page \pageref{pg:empty} in this document)
		\item adding space, changing fonts and text alignment locally with existing commands: \\ 
		\verb|One {\Large{word}} appears large|
	\end{itemize}
	
\end{itemize}

\clearpage

{\centering 
	
	\textbf{Here are a few simple examples \\ Notice and explore the numbering of the examples linked to Sections}
	
}
\begin{examples}
	This is some text with the default \LaTeX command. \par
	And this sentence calls the modified \myLaTeX command. \par
	(Notice the added space after the logo.)
\end{examples}

\begin{wrapfigure}{r}{0.3333\linewidth}
	\boxedfig{lines.eps}
	\caption{Two lines plot in a box}
	\label{fig:logo}
\end{wrapfigure}


\begin{examples} 
	\par
	Two calls of the hello environment:
	\begin{lstlisting}
	\begin{hello}
	nice to meet you.
	\end{hello}
	
	\begin{hello}[Bob]
	glad you could make it.
	\end{hello}
	\end{lstlisting}
	
	Output:\\
	\begin{hello}
		nice to meet you.
	\end{hello}
	\begin{hello}[Bob]
		glad you could make it.
	\end{hello}
\end{examples}



\begin{examples}
	\begin{lstlisting}
		\renewcommand*{\figurename}{Fig.}
		
		\newcommand{\boxedfig}[1]{%
		\setlength{\fboxsep}{5pt}%
		\setlength{\fboxrule}{3pt}%
		\fbox{\includegraphics[width=\linewidth]{#1}}%
		}
	\end{lstlisting}
	
	Called as:
	\begin{lstlisting}
		\begin{wrapfigure}{r}{0.3333\linewidth}
			\boxedfig{lines.eps}
			\caption{Two lines plot in a box}
			\label{fig:logo}
		\end{wrapfigure}
	\end{lstlisting}
\end{examples}

\begin{examples}
	\verb|\newcommand{\hi}[2][world]{\noindent Hello #1 and #2}| \\
	called as 	\verb|\hi[Marie][Stephen]| and as \verb|\hi{Stephen}|\\
	
	\hi[Marie]{Stephen}
	
	\hi{Stephen}
\end{examples}

\textcolor{red}{NB!} Asterisks in commands definitions and per cent signs at the end of lines are safety checks to prevent arguments accedentally containing blank lines or \verb|\par|.

\bigskip
\textcolor{red}{NB!} Renewing commands that have \verb|@| in their name requires:
\begin{lstlisting}
\makeatletter
\renewcommand*{\verbatim@font}{\ttfamily\footnotesize}
\makeatother
\end{lstlisting}
This sort of redefinition cannot be used in .sty files.

\clearpage

\section{Internal and external links}\label{sec:links}

The main package to allow cross-referencing is \verb|\usepackage{hyperref}|.

Types of links:

\begin{itemize}
	\item Internal links (inc. to individual words): \verb|\label{sec:links} ... \ref{sec:links}|
	
	In Section~\ref{sec:links} we used \ldots
	
	\item Links to local files: \\ \verb|\href{run:./pics/Pym_2020_translation_solutions_ES>EN.pdf}{Pym's paper (2020)}| 
	
	See \href{run:./pics/Pym_2020_translation_solutions_ES>EN.pdf}{Pym's paper (2020)}
	
	
	\item Web addresses: \verb|\href{https://en.wikipedia.org/wiki/LaTeX}{Wiki on Latex}| \\
and \verb|\url{https://en.wikipedia.org/wiki/LaTeX}| 

	This is what Wikipedia says about Latex: \href{https://en.wikipedia.org/wiki/LaTeX}{Wiki on Latex} or with visible address \url{https://en.wikipedia.org/wiki/LaTeX}

\end{itemize}

\textbf{Custom colours for each type of links (seems to be a paper-friendly solution)}

\begin{lstlisting}[breaklines]
	\hypersetup{
		colorlinks=true,
		linkcolor=blue,
		filecolor=magenta,      
		urlcolor=cyan,
	}
\end{lstlisting}

% ; load early in the preamble as some packages complain 
To refer back to a particular word/phrase in the document, use: 
\begin{lstlisting}[breaklines]
\hypertarget{wd:cats}{where_to_return} ...
 \hyperlink{wd:cats}{word_to_make_clickable}
\end{lstlisting}

\thispagestyle{empty}\label{pg:empty}

In Section~\ref{sec:own}, we talked about some \hyperlink{wd:random}{cats}.

\bigskip

Hide all the clickables (good for printing on paper, but not for an e-document): \\
\verb|\usepackage[hidelinks]{hyperref}|

\section{Templates and big projects}
\begin{itemize}
	\item Many publication venues provide their own \LaTeX templates. It makes sense to keep the original source file (.tex) in the project folder for reference. 
	\item Notice lack of extensions in imports.
\end{itemize}

A zipped \LaTeX~template may include the following files (Look into the source code to see another method to produce columns and control their size and positioning):

\medskip

\begin{minipage}[c]{0.65\linewidth}
	\begin{tcolorbox}[title={Computational Linguistics by MIT Press Journals}]
		\begin{description}
			\item[clv3.cls] class file, used in \verb|documentclass{clv3}|
			\item[alocal.sty] style file (a patch for ArabTeX package) to be imported with \verb|\usepackage{alocal}| if necessary
			\item[compling.bst] bibliography style file, used as \verb|\bibliographystyle{compling}|
			\item[compling\_style.bib] example file with bibliography records 
			\item[COLI-manual3.tex] main source code 
		\end{description}
	\end{tcolorbox}
\end{minipage} 
\hfill  % comment out space fill if you would like to put them side by side
\begin{minipage}[c]{0.3\linewidth}
	\begin{tcolorbox}[title={RANLP conference}]
		\begin{itemize}
			\item ranlp2021.sty
			\item acl\_natbib.bst
			\item anthology.bib
			\item ranlp2021.tex
		\end{itemize}
	\end{tcolorbox}
\end{minipage}

\subsection{Principles of working on a multi-part book (like a thesis)}

\begin{itemize}
	\item Adapting an existing template maybe easier than setting up your own from scratch.
	\item Using a template implies familiarity with the packages and commands.
\end{itemize}

\textbf{Specificity of a multi-part book-like project: What is different?}

\begin{itemize}
	\item parts are typeset in separate files which are imported by the main code which defines overall parameters -- e.g. thesis.tex importing 
	
	\begin{lstlisting}
	% this is source code for one of the sessions in Digital Skills for Research Workshop (EMTTI, University of Wolverhampton)
% March 2022, Maria Kunilovskaya (mkunilovskaya@gmail.com)

%--------------------
% Preamble

% Declare the type of document
% -------------------

\documentclass[a4paper,12pt]{article} % other options: [,twocolumn,leqno]{report, book, beamer} 

%--------------------
% Import packages
% -------------------

\usepackage[utf8]{inputenc}  % specify the encoding
\usepackage{hyperref} % allow cross-referencing
\usepackage{geometry} % set the layout
\geometry{
	a4paper,
	total={170mm,257mm},
	left=20mm,
	top=20mm,
}
\usepackage{xcolor} % alow color for text
\usepackage{tcolorbox} % make boxes
\usepackage{multicol} % columns

\usepackage{graphicx} % insert pictures
%\graphicspath{{images/}{pics/}}  % from folders with .png, .jpj, .gif, .eps

%--------------------
% Define new commands and settings
% -------------------

% TeX logo as defined by Donald Knuth in the TeXbook (1984)
\def\TeX{{\rm T\kern-.1667em\lower.5ex\hbox{E}\kern-.125emX }}
\newcommand{\llogo}{\LaTeX }

\setlength\parindent{0pt} % don't indent new paragraphs

%--------------------
% Setup the title of the document
% -------------------

\title{\vspace{-4em} Digital Skills for Research}
\author{Maria Kunilovskaya\thanks{thanks to the RGCL for the opportunity to brush up and systematise these}}
\date{02 March - 25 March, 2022}

\begin{document}
	
%remove numbering from the first page
\clearpage\maketitle
\thispagestyle{empty}	
\maketitle

\vspace{-2em}

\section{{\color{red}\TeX and \LaTeX}}

\subsection{Week 1. Setup and Start}\footnote{to compile pdfs download the necessary support files from the respective folders, not just the linked .tex}
	Session 1. Why \TeX and first doc (\href{https://canvas.wlv.ac.uk/courses/33429/files/folder/latex_mendeley_github/w1-3_latex?preview=4622172}{pdf}, \href{https://github.com/kunilovskaya/dskills_workshop/blob/main/w1_latex_basics/session1.tex}{class tex}, \href{https://github.com/kunilovskaya/dskills_workshop/blob/main/w1_latex_basics/task1.tex}{task})
		\begin{itemize}
			\item What's \TeX (distributions, editors, engines, formats, templates)
			\item Overleaf account and project
			\item First document (class, packages, layout, title, sections, margins, columns)
		\end{itemize} 
	Session 2. Text and math (\href{https://canvas.wlv.ac.uk/courses/33429/files/folder/latex_mendeley_github/w1-3_latex?preview=4623463}{pdf}, \href{https://github.com/kunilovskaya/dskills_workshop/blob/main/w1_latex_basics/session1.tex}{class tex}, \href{https://github.com/kunilovskaya/dskills_workshop/blob/main/w1_latex_basics/task2.pdf}{task})
		\begin{itemize}
			\item Text formatting
			\item Special characters
			\item Math
		\end{itemize}

\subsection*{Week 2. Environments and Customisation}
	Session 3. Tables and figures (\href{https://github.com/kunilovskaya/dskills_workshop/blob/main/w2_latex_frills/session3.tex}{class tex}, \href{https://github.com/kunilovskaya/dskills_workshop/blob/main/w2_latex_frills/task3.tex}{task})
			\begin{itemize}
				\item Environments
				\item Tables
				\item Graphics and drawing
			\end{itemize}
	Session 4. Customisation, cross-referencing and templates (\href{https://github.com/kunilovskaya/dskills_workshop/blob/main/w2_latex_frills/session4.tex}{class tex}, \href{https://github.com/kunilovskaya/dskills_workshop/blob/main/w2_latex_frills/task4.tex}{task})
			\begin{itemize}
				\item Own commands
				\item Internal and external links
				\item Templates and big projects
			\end{itemize}

\subsection*{Week 3. Beamer and Bibs}
	Session 5. Presentations and posters (\href{https://github.com/kunilovskaya/dskills_workshop/blob/main/w3_bibs_beamer/session5.tex}{class tex}, \href{https://github.com/kunilovskaya/dskills_workshop/blob/main/w3_bibs_beamer/task5.tex}{task})
	\begin{itemize}
		\item Beamer themes: layout and colour
		\item Slides-specific commands
		\item Producing posters and own .sty 
	\end{itemize}

\newpage

	Session 6. Bibliographies (\href{https://canvas.wlv.ac.uk/courses/33429/files/folder/LaTeX\%20and\%20Mendeley\%20workshop/w1-3_latex?preview=4648015}{class pdf}, \href{https://github.com/kunilovskaya/dskills_workshop/blob/main/w3_bibs_beamer/task6.tex}{task})
	\begin{itemize}
		\item Basics on referencing styles
		\item .bib files and bibliography engines
		\item Get References section
	\end{itemize}	

\section{{\color{red}Week 4. Mendeley and Github}}
	Session 7. Reference management: Mendeley (\href{https://canvas.wlv.ac.uk/courses/33429/files/folder/LaTeX\%20and\%20Mendeley\%20workshop/w1-3_latex?preview=4661728}{class pdf}, \href{https://canvas.wlv.ac.uk/courses/33429/files/folder/LaTeX\%20and\%20Mendeley\%20workshop/w1-3_latex?preview=4661729}{task})
	\begin{itemize}
		\item Basic uses and setting up
		\item Integration (word processor, browser, project)
		\item Library use and maintenance 
		% (+ note-taking tools)
	\end{itemize}%
	Session 8. Version control and collaboration: Git and GitHub (\href{https://canvas.wlv.ac.uk/courses/33429/files/folder/LaTeX\%20and\%20Mendeley\%20workshop/w1-3_latex?preview=4661911}{class pdf}, \href{https://canvas.wlv.ac.uk/courses/33429/files/folder/LaTeX\%20and\%20Mendeley\%20workshop/w1-3_latex?preview=4661912}{task})
	\begin{itemize}
		\item Keeping track of changes
		\item Local and remote, push and pull, auth
		\item Markdown and arranging repos
	\end{itemize}

\begin{center}
	
\begin{tcolorbox}[width=\textwidth, colback={blue!10!white}, title={\textbf{Learning strategies behind the workshop design (in the order of importance)}}, colbacktitle=blue!30!white, coltitle=black]
	\begin{itemize}
		\item distributed learning: building skills over time VS cramming all at once
		
		\begin{center}
			\includegraphics[width=40 mm]{pics/all at once.png} \hspace{2cm}
			\includegraphics[width=30 mm]{pics/a little every day.png}
		\end{center}
	
		\item no-stakes quizzes and retrieval practice (frequent creative engagement with the learnables)
		\item alternating practice (new skills interspersed with old material)
		\item organisation of learning effort and stimulus to make in NOW
	\end{itemize}

\bigskip

+ (hopefully) motivation from social interaction and useful personal insights
	
\end{tcolorbox}%

\bigskip

\begin{tcolorbox}[width=\textwidth, colback={yellow!40!white}, title={\textbf{Housekeeping}}, colbacktitle=yellow!60!white, coltitle=black]
	
	\begin{multicols}{2}

		\begin{itemize}
			\item a class content file and a practical task for each session
			\item Canvas: \href{https://canvas.wlv.ac.uk/courses/33429/files/folder/latex_mendeley_github}{Materials}; \href{https://wlv.instructure.com/courses/33429/pages/latex-and-mendeley-workshop}{Schedule} \\ Wed, Fri 10am-11am, live sessions
			\item GitHub: \href{https://github.com/kunilovskaya/dskills_workshop}{Source code for all sessions} 
			\item Send, or share with, me your source code (m.kunilovskaya@wlv.ac.uk) and track your progress in the course \href{https://docs.google.com/document/d/1XKaCl3-tRkNfoy1w6dSrqrQBBLLrpnnLBt1zi5KXec0/edit?usp=sharing}{Achievement and Attendance Tracker} to earn a certificate and ascertain your level: confident user, familiar with LaTeX, basic skills
			\item whatsup: +44 7926 446507 $\rightarrow$
		\end{itemize}
	
		\columnbreak
		
		\centering
		 \includegraphics[width=40mm]{pics/maria_ku_whatsup_contact} \\
		Naming convention for any submissions: \\
		ID\_task\#.pdf \\
		where 
		ID is your unique name and 
		\# is the number of the training session
	 
	\end{multicols}
\end{tcolorbox}%

\bigskip

\begin{tcolorbox}[width=\textwidth, colback={white}, title={\textbf{Recommended resources}}, colbacktitle=white, coltitle=black]
	\begin{itemize}
	\item \textbf{LaTeX}: \href{https://tug.org/begin.html}{\TeX UsersGroup} and \href{https://www.overleaf.com/learn/latex/Learn_LaTeX_in_30_minutes}{Overleaf} tutorials
	\item \textbf{GitHub}: \href{https://git-scm.com/book/en/v2}{Pro Git book by Scott Chacon and Ben Straub} and \href{https://docs.github.com/en/get-started}{Get started from github.com}
	\item \textbf{Mendeley}: \href{https://www.mendeley.com/guides}{Official Guides}
	
\end{itemize}
	
\end{tcolorbox}%

\end{center}

\end{document} 
	\include{tables}
	\include{chapters/dedication}
	\include{chapters/1_intro}
	\end{lstlisting}
	
	\item \verb|\documentclass[12pt,a4paper,twoside,openright]{report}|
	\item \textcolor{red}{additional elements in text}:
	
	\begin{itemize}
		\item \verb|\usepackage{epigraph} ... \epigraph{}{}|
		\item headers-footers: 
			\begin{lstlisting}
	\usepackage{fancyhdr}
	\pagestyle{fancy}
	\fancyhead{}
	% RO/LE: position of text on odd/even pages
	\fancyhead[RO,LE]{Thesis Title}	
			\end{lstlisting}
			
		\item \verb|\usepackage{todonotes} ... \todo[inline]{...}|
		
		\item a glossary to collect all abbreviations and acronyms 
		\todo[inline]{This comment is generated by a todo[inline] command: Do I need a glossary?}
		\begin{lstlisting}
	\usepackage[acronym,automake]{glossaries}
	\makeglossaries
		\end{lstlisting}
		
		\item code listings using verbatim \verb|usepackage{listings}|

		\begin{lstlisting}
	\lstinputlisting[language=Python]{argpars.py}
		\end{lstlisting}
		
	\lstinputlisting[language=Python, firstline=2, lastline=15, caption=Types of arguments for a Python script from a file]{argpars.py}

		\begin{lstlisting}[language=Python, caption=Looping thru all folders under 'root']
		for path, dirs, files in os.walk(args.root):
		last_folder = os.path.abspath(path).split('/')[-1]
		for i, file in enumerate(files):
		filepath = path + os.sep + file
		
		\end{lstlisting}

	\end{itemize}
\end{itemize}

\lstlistoflistings


\section*{Task 4. Create a document based on a template}
\label{task}
\addcontentsline{toc}{section}{Task 4. Create a document based on a template}

\begin{tcolorbox}[width=\textwidth, colback={yellow!40!white}, title={You can complete any of the following tasks for this session. Each involves selecting a template and exploring the commands it uses by modifying it. Acknowledge the original template (add a link to it)in your source code}, colbacktitle=yellow!60!white, coltitle=black]
	\begin{itemize}
		\item Produce a CV using one of the templates \\ \href{here}{https://www.latextemplates.com/cat/curricula-vitae}
		\item Prepare a mock submission to RANLP using their \href{https://www.overleaf.com/latex/templates/instructions-for-ranlp-2021-proceedings/snyphxfdqcpz}{Overleaf template}
		\item Explore and set up a multi-chapter phd/master thesis template based on what is \href{https://github.com/snim2/phdtemplate}{provided by the University of Wolverhampton} and (most recently, in Oct 2021) adapted by \href{https://github.com/TharinduDR/Thesis/}{Tharindu Ranasinghe}
	\end{itemize}
	
\end{tcolorbox}%

\end{document}